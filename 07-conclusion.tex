\section{Conclusion}

In this document, we analyzed posts from Facebook and Twitter for politicians and athletes by visualizing them for different aspects, namely (i) contribution over time (ii) summary of posts (iii) identify topics from posts and (iv) impact of posts. So, visualization helps to better analyze the data and reveal some hidden insights of them. 


Moreover, we provided detailed description about the topics models, such as Latent Dirichlet Allocation and Non-negative Matrix Factorization that are used for extracting topics from users posts. This approach aims to have an interesting view about the themes (or topics) a user is discussing in social network. According to find the similarity of a users posts, the cosine similarity metric is used. In this way, 
we had a view about how popular is a similar or same post in a social network. 


As a future work, it would be interesting to find similar posts among users on a socials network. This will aim to observe if the users are interested in talking about similar topics and then it is quite possible that they are interested in similar things. Another interesting direction is to take into account more values for the impact of posts such as comments, shares, retweets etc. By looking, for instance the comments of a user, we can gain an interesting view about the sentiment of the post i.e if it positive or negative. Lastly, it would also be interesting to investigate other approaches to visualize posts of a user as well as use the different visualizations together.

