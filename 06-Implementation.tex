\section{Prototype implementation}

In this section we provide a detailed description of our framework, the data collection and the technologies that used in the front and back ends.


\subsection{Framework}

Our web application is based on Django\footnote{\url{https://www.djangoproject.com/}}, which is a free and open-source web framework, written in Python, and follows the model–view–controller (MVC) architectural pattern.

\subsection{Data Collection}

Our data collection consists of real-world data from Facebook 
and Twitter and focus on 28 public persons such as, politicians 
and athletes because they tend to post the same content on Twitter and Facebook more than a normal user.

\subsubsection{Retrieve data from Facebook}
 
 %TO DO

\subsubsection{Retrieve data from Twitter}

The data was obtained by quering the timeline API of Twitter with the username of each person related to politicians and athletes. For this procedure we used the Tweepy\footnote{\url{http://www.tweepy.org/}}, which is a Python library for accessing the Twitter API. We were able to collect a fixed number of tweets because Twitter only allows access to a users most recent 3240 tweets. The attributes of the data along with their definitions
are displayed in Table \ref{table:attrib_des}.

\begin{table}[ht] 
\caption{Description of the attributes of Twitter data} 
\centering  
\begin{tabular}{c | c} 
\hline\hline 
Attribute & Description \\ [0.5ex] 
\hline 
ID & The id of the twitter post \\ 
date & The date when the tweet was posted \\ 
text & The text of the tweet\\ 
likes & The number of likes of a tweet \\ [1ex]  
\hline  
\end{tabular} 
\label{table:attrib_des}
\end{table} 

\subsection{Data visualization}
\subsubsection{Contribution over time}
\subsubsection{Summary of posts}
\subsubsection{Topic models}
\subsubsection{Impact of posts}


%%% Local Variables: 
%%% mode: latex
%%% TeX-master: "isae-report-template"
%%% End: 