\section{User contribution visualization}\label{sec:uservisualization} 

As a result of our research about social networks and social network contributions we came up with four different aspects which we wanted to visualize during our work. These aspects are: the amount of contribution over time, a summary of posts, topic models and the impact of one post in different networks. For each of this aspect there are different reasons why it makes sense to visualize it. 


If we visualize the contribution over time we can identify interesting points in time. For example if there is a month, where a user wrote more posts than usual we can assume that this month was important for this user and we can start to do more research about what happened at this time. Also an interactive visualization about the contribution over time enables a user to read post from a specific time in the past. An example here could be  a student that want to find out about the social network behavior of politicians during a election campaign. He knows the month which are important for the election. A visualization about the contributions of politicians over time with the possibility to click on one month and get all the post of this month, would make it easy for the student to get his desired posts. 


With the summary of posts we want to show about what a user is talking in the easiest way possible. A presentation like this makes it possible  to get a quick overview about the posts from a user. If for example we find, with the visualization of the amount of contribution, an interesting month  a summary about the posts can help to identify why the user posted so much in this month.


Another aspect are the topic models, which give us an overall overview about which topic (themes) a user is posting. Here it can be interesting, that once found the topics, to read the posts which belong to a special topic. For example if we analyze the topics posted by a politician and we find one topic which contains the words "health, insurance, care", we can assume that the politician talks  in some of his posts about health-care. We then can get the posts which belong to  this topic and read what this politician has to say about  health-care. 


Our last aspect is maybe not as much important for a normal user than for the writer itself. Showing the different impacts (number of likes, shares, comments) a post has in different social networks enables the writer of the post to see in which social network he has more followers and people who agree with him. Also, they can compare the impact of posts in the same social network to analyze if one post has more likes than the other and find out why this could be.  Especially, for people who are promoting themselves in social networks it is important to see how much impact there posts have. They need to know what it is that make people like or share their posts, because just then friends of these people see their posts as well and their message gets spread. 




